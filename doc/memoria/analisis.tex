%%%%%%%%%%%%%%%%% TOMA DE REQUISITOS %%%%%%%%%%%%%%%%%%%%%%%5
\section{Especificación de requisitos del sistema}

\paragraph{}
Para la creación de cualquier producto software, es necesario establecer las distintas condiciones y necesidades que
ha de satisfacer. Seguiremos un esquema que nos permita describir los requisitos de una forma metódica y racional.
%\paragraph{}

\subsection{Requisitos de interfaces externas}

\paragraph{}
En este apartado se describirá los requisitos de conexión del software y el hardware, así como la interfaz de usuario.

\paragraph{}
La conexión entres el software y el hardware se encarga la librería \emph{SDL}, mediante el wrapper \emph{Pygame} para 
el lenguaje de programación \emph{Python}. Por lo que al ser un sistema preestablecido, no será necesario realizar el diseño,
ni el análisis, sólo haremos uso de él.

\paragraph{}
Así que pasamos a definir la interfaz entre el usuario y el videojuego. Todas las ventanas de la aplicación tendrán una 
resolución de 800x600 píxeles, siendo posible establecer el modo de pantalla completa \footnote{El modo de pantalla completa
se podrá establecer a través del menú de opciones}. A continuación se distinguen las distintas
ventanas que el usuario encontrará en el sistema:

\begin{description}
    \item \textbf{Ventana de introducción} En esta primera ventana se mostrará únicamente el logotipo del juego, situando al usuario en
    contexto para introducirlo en la ejecución de la aplicación.
    
    \item \textbf{Ventana de menú principal} La ventana del menú principal (figura \ref{fig:boceto-menu-principal})
    muestra el menú de inicio de \emph{Zycars}, así como 
    todas las opciones generales del juego disponibles, que son las siguientes:
        \begin{itemize}
            \item Carrera Rápida
            \item Campeonato
            \item Contrarreloj
            \item Opciones
            \item Créditos
            \item Salir
        \end{itemize}
        En este menú y en los siguiente que se describan se usará el ratón para navegar por ellos y solo será necesario
        hacer click sobre la opción deseada para acceder a ella.

        \figura{imagenes/analisis/boceto_menu_principal.png}{scale=0.5}{Análisis: Boceto del menú principal}{fig:boceto-menu-principal}{H}
    
    \item \textbf{Ventana de opciones} Desde la ventana de opciones (figura \ref{fig:boceto-opciones}) se podrán
    modificar las distintas características de la 
    configuración del juego como el audio o controles. Esta ventana se podría dividir en tres partes diferencias que se indican
    a continuación:
        \begin{itemize}
            \item Opciones de audio: podremos modificar el volumen de efectos de sonido y música del juego. También estará la opción
            de silenciar cualquier tipo de sonido.
            
            \item Opciones de pantalla: a través de esta ventana podremos activar o desactivar el modo de pantalla completa.
            
            \item Opciones de control: en esta ventana podremos modificar los controles del juego. Tanto de dirección, lanzamiento
            de ítems y pausa del juego.
        \end{itemize}

        \figura{imagenes/analisis/boceto_opciones.png}{scale=0.5}{Análisis: Boceto del menú de opciones}{fig:boceto-opciones}{}

    \item \textbf{Ventana de créditos} En esta pantalla (figura \ref{fig:boceto-creditos})se mostrarán los creadores de \emph{Zycars}. Tendremos a nuestra 
    disposición un botón para volver al menú principal.

        \figura{imagenes/analisis/boceto_creditos.png}{scale=0.5}{Análisis: Boceto de la pantalla de créditos}{fig:boceto-creditos}{}
    
    \item \textbf{Ventana de selección de personaje} Esta ventana (figura \ref{fig:boceto_menu_personaje}) será compartida por los tres modos de juego disponibles. En ella 
    podremos elegir al personaje que desearemos controlar a lo largo de las carreras. De estos jugadores se nos mostrarán sus 
    distintas habilidades.

        \figura{imagenes/analisis/boceto_menu_personaje.png}{scale=0.5}{Análisis: Boceto del menú de selección de personaje}{fig:boceto_menu_personaje}{H}
    
    \item \textbf{Ventana de selección de circuito} Ventana (figura \ref{fig:boceto_menu_circuito}) compartida por el modo carrera rápida y contrarreloj. En esta ventana
    deberemos elegir el circuito en el que deseamos competir. Se nos mostrará una imagen de cada circuito que seleccionemos.

        \figura{imagenes/analisis/boceto_menu_circuito.png}{scale=0.5}{Análisis: Boceto del menú de selección de circuito}{fig:boceto_menu_circuito}{H}
    
    \item \textbf{Ventana de selección de campeonato} Ventana muy similar a la descrita anteriormente. En ella se nos mostrarán
    todos los circuitos de cada campeonato disponible. Pero al contrario que la anterior en esta ventana elegiremos el campeonato 
    del circuito seleccionado en el momento.
    
    \item \textbf{Ventana de juego} Ventana principal de todo el juego (figura \ref{fig:boceto_juego}). Mostrará la carrera actual que se esté disputando, así como
    los distinto marcadores aclaratorios sobre el estado de la carrera, como
    pueden ser posiciones de los jugadores, ítem actual y
    tiempos de carrera. Según la tecla indicada en los controles del menú de opciones (ESC o p) se podrá acceder al menú
    de pausa del juego.

        \figura{imagenes/analisis/boceto_juego.png}{scale=0.5}{Análisis: Boceto de ventana de juego}{fig:boceto_juego}{}
    
    \item \textbf{Ventana de pausa} Únicamente accesible desde la ventana de juego (figura \ref{fig:boceto_pausa}). Esta nos permitirá detener el juego en curso, 
    siendo posible reanudar el juego, reiniciar el mismo o volver al menú principal.

        \figura{imagenes/analisis/boceto_pausa.png}{scale=0.5}{Análisis: Boceto de menú de pausa}{fig:boceto_pausa}{}
    
    \item \textbf{Ventana de posiciones de carrera} Ventana mostrada (figura \ref{fig:boceto_resultados_carrera}) al terminar alguna de las carreras. En ella nos muestra el 
    resultado de la última carrera disputada. Nos permite continuar al siguiente circuito, en el caso del modo campeonato, o seguir
    hacia el menú principal, en el modo carrera rápida. También se nos permite
    reiniciar la última carrera disputada.

        \figura{imagenes/analisis/boceto_resultados_carrera.png}{scale=0.5}{Análisis: Boceto de la ventana de posiciones}{fig:boceto_resultados_carrera}{H}
        
    \item \textbf{Ventana de posiciones de campeonato} Ventana mostrada en el modo campeonato tras la ventana de posiciones 
    de carrera, en ella se nos muestra las posiciones de los competidores en el campeonato actual.
    
    \item \textbf{Ventana de tiempos de contrarreloj} Ventana (figura \ref{fig:boceto_tiempos}) mostrada al completar algún circuito en el modo contrarreloj. En ella
    se nos muestran los distintos tiempos conseguidos a lo largo del circuito y
    se nos indicará si hemos batido algún récord.
    Esta ventana nos permite continuar hacia el menú principal, así como reiniciar el circuito disputado.

        \figura{imagenes/analisis/boceto_tiempos.png}{scale=0.5}{Análisis: Boceto de la ventana de tiempos}{fig:boceto_tiempos}{H}
    
\end{description}

\subsection{Requisitos funcionales}

\paragraph{}
Los requisitos funcionales que el sistema debe ofrecer son los siguientes:

\begin{itemize}
    \item Salir de la aplicación desde cualquier ventana.
    \item Seleccionar los distintos modos de juego.
    \item Permitir al jugador competir contra coches dirigidos por el ordenador.
    \item Modificar la configuración (audio, pantalla y controles) del juego.
    \item Pausar el juego.
    \item Seleccionar uno de los jugadores propuestos.
    \item Seleccionar cualquiera de los circuitos disponibles.
    \item Seleccionar cualquiera de los campeonatos disponibles.
    \item Reiniciar cualquier carrera una vez terminada o en curso.
    \item Reiniciar cualquier campeonato una vez terminado o en curso.
    \item Lanzamientos de ítems durante cualquier carrera.
    \item Extender el juego mediante la incorporación de nuevos circuitos y personajes.
\end{itemize}

\paragraph{}
Los distintos tipos de jugadores son:

\begin{itemize}
    \item \textbf{Humano}: es el controlado por una persona
    \item \textbf{Máquina}: controlado por el ordenador.
\end{itemize}

\paragraph{}
Existes tres modos de juego:

\begin{itemize}
	\item \textbf{Carrera rápida}: consiste en la realización de un único circuito, compitiendo contra coches dirigidos 
    por el ordenador.
	\item \textbf{Campeonato}: el jugador competirá contra coches dirigidos por el ordenador a lo largo de 4 carreras, 
    en las que obtendrá una puntuación
	según la posición obtenida en cada una de las carreras. El ganador será el que mejor puntuación haya conseguido al concluir
	el campeonato.
	\item \textbf{Contrarreloj}: en este modo de juego, el jugador competirá solo, con el fin de mejorar las marcas de tiempo de cada uno de los 
	circuitos.
\end{itemize}


\subsection{Requisitos de rendimiento}

\paragraph{}
El rendimiento de la aplicación debe ser tal que permita un desempeño agradable de juego. 

\begin{itemize}
    \item Por lo que la respuesta a las acciones realizadas por el usuario deben ser respondidas lo más rápido posible,
    sacrificando en el caso de que sea necesario el consumo de la memoria principal.
    
    \item La inteligencia artificial debe estar optimizada de forma que no se ralentice la partida en el tiempo dedicado a los
    cálculos necesarios para tomar decisiones.
\end{itemize}
    

\subsection{Restricciones de diseño}

\paragraph{}
Como comento en uno de los puntos del apartado anterior el tiempo de respuesta tiene que primar sobre el consumo de 
memoria principal o secundaria. Esta será la principal restricción de diseño que tendrá nuestra aplicación.

\paragraph{}
Los videojuegos están pensados como aplicación principal, de forma que no tenga que compartir recursos con otros procesos,
por lo que se permitirá que consuma muchos recursos del sistema.

\subsection{Requisitos del sistema software}

\paragraph{}
La aplicación deberá cumplir los siguiente requisitos del sistema:

\begin{itemize}
    \item Deberá ser multiplataforma, al menos en los siguiente sistemas:
    \begin{itemize}
        \item \textbf{Microsoft Windows}: realizando las pruebas sobre la versión Windows 7.
        \item \textbf{GNU/Linux}: usando la distribución Ubuntu 10.10 como principal sistema para pruebas.
    \end{itemize}
    
    \item El código con el que se desarrolle la aplicación no debe ser
    dependiente del sistema en el que se desarrolle.
    
    \item El código debe ser mantenible y fácilmente ampliable para futuras versiones.
\end{itemize}

%%%%%%%%%%%%%%%% MODELO DE CASOS DE USO %%%%%%%%%%%%%%%%%%%%%%%
\section{Modelo de casos de uso}

\paragraph{}
Para describir los distintos comportamientos que tendrá el sistema, usaremos el lenguaje de modelado de sistemas \emph{UML}; que
representa los requisitos funcionales del sistema, centrando en que hace y no cómo lo hace.

\subsection{Diagrama de los casos de uso}

\paragraph{}
En primer lugar mostramos el modelo de casos de uso (figura \ref{fig:diagrama_casos_uso}), que representa la funcionalidad completa de la aplicación. Se ha usado el 
siguiente esquema:

\begin{enumerate}
    \item Identificar los usuarios del sistema y los roles que pueden tener.
    \item Para cada rol, identificar las distintas formas de interactuar en el sistema. En el caso de \emph{Zycars} existe
    un único rol de acceso a la aplicación, por lo que la especificación del usuario será única.
    \item Creación de los casos de uso para todos los objetivos que queramos cumplir.
    \item Estructurar dichos casos de uso.
\end{enumerate}

\figura{imagenes/analisis/diagrama_casos.png}{scale=0.56, angle=90}{Análisis: Diagrama de casos de uso}{fig:diagrama_casos_uso}{}

\subsection{Descripción de los casos de uso}

\paragraph{}
A continuación pasamos a la descripción de cada uno de los casos de uso, para la cual usaremos una notación forma usando plantillas.
El texto debe ser legible y comprendido por un usuario que no sea experto.

\subsubsection{Caso de uso: Menú principal}

\begin{description}
    \item[Caso de uso] Menú principal
    \item[Descripción] Se muestra el menú principal de la aplicación, donde es
    posible elegir uno de los modos de juego disponibles
    o acceder al menú de opciones.
    \item[Actores] Usuario
    \item[Precondiciones] Ninguna
    \item[Postcondiciones] Ninguna
    \item[Escenario principal] $\quad$
        \begin{enumerate}
            \item El usuario inicia la aplicación
            \item El sistema muestra el menú principal del juego en pantalla.
            \item El usuario selecciona la opción \textbf{carrera rápida}, \textbf{campeonato} o \textbf{contrarreloj}.
            \item El sistema inicia el modo de elección de personaje.
        \end{enumerate}
    \item[Extensiones --- flujo alternativo] $\quad$
        \begin{description}
            \item[*a ] El usuario cierra la ventana de la aplicación y sale de la aplicación
            
            %\item[2a ] El usuario seleccion la opción \textbf{campeonato}.
            %\begin{enumerate}
            %    \item El sistema inicia el modo de elección de personaje
            %\end{enumerate}
            
            %\item[2b ] El usuario seleccion la opción \textbf{contrarreloj}.
            %\begin{enumerate}
            %    \item El sistema inicia el modo de elección de personaje
            %\end{enumerate}
            
            \item[2a ] El usuario selección la opción \textbf{opciones}.
            \begin{enumerate}
                \item El sistema inicia las opciones
            \end{enumerate}
            
            \item[2b ] El usuario selección la opción \textbf{créditos}.
            \begin{enumerate}
                \item El sistema inicia la opción créditos
            \end{enumerate}
            
            \item[2c ] El usuario selección la opción \textbf{salir}.
            \begin{enumerate}
                \item El sistema sale de la aplicación.
            \end{enumerate}
            
        \end{description}
\end{description}

\subsubsection{Caso de uso: Elegir personaje}

\begin{description}
    \item[Caso de uso] Elegir personaje
    
    \item[Descripción] El usuario desea seleccionar el personaje con el que competirá en el juego, ya sea carrera rápida, 
    campeonato o contrarreloj.
    
    \item[Actores] Usuario
    
    \item[Precondiciones] Ninguna
    
    \item[Postcondiciones] Se selecciona un jugador y se almacenará en la configuración de juego.
    
    \item[Escenario principal] $\quad$
        \begin{enumerate}
            \item El usuario desea seleccionar al personaje que usará en el juego.
            \item El sistema muestra la pantalla de elección de personaje y carga todos los personajes disponibles.
            \item El usuario selecciona un personaje de la lista y pulsa la opción aceptar.
            \item El sistema almacena en la configuración el jugador seleccionado.
            \item El sistema pasa a selección de circuito%, si la precondición es \textbf{carrera rápida} o \textbf{contrarreloj}
        \end{enumerate}
    \item[Extensiones --- flujo alternativo] $\quad$
        \begin{description}
            \item[*a ] El usuario cierra la ventana de la aplicación y sale de la aplicación
            
            \item[*b ] El usuario pulsa la opción cancelar.
            \begin{enumerate}
                \item El sistema vuelve al menú principal
            \end{enumerate}
            
            \item [5a ] El sistema pasa a selección de campeonato.
            
        \end{description}
\end{description}

\subsubsection{Caso de uso: Elegir circuito}

\begin{description}
    \item[Caso de uso] Elegir circuito
    \item[Descripción] El usuario desea seleccionar el circuito en el que desea competir.
    \item[Actores] Usuario
    \item[Precondiciones] El usuario ha elegido previamente en el menú principal una de las opciones: \textbf{carrera rápida} 
    o \textbf{contrarreloj} y ya ha seleccionado un personaje.
    
    \item[Postcondiciones] Se selecciona un circuito y se almacena en la configuración del juego.
    \item[Escenario principal] $\quad$
        \begin{enumerate}
            \item El usuario desea seleccionar el circuito en el que competir.
            \item El sistema muestra todos los circuitos disponibles (icono e imagen) para cada campeonato.
            \item El usuario selecciona un circuito y pulsa aceptar
            \item El sistema almacena el circuito en la configuración del sistema.
            \item El sistema accede a la pantalla de jugar carrera.
        \end{enumerate}
    \item[Extensiones --- flujo alternativo] $\quad$
        \begin{description}
            \item[*a ] El usuario cierra la ventana de la aplicación y sale de la aplicación
            \item[*b ] El usuario pulsa la opción cancelar.
            \begin{enumerate}
                \item El sistema vuelve a la selección de personaje.
            \end{enumerate}
        \end{description}
\end{description}

\subsubsection{Caso de uso: Elegir campeonato}

\begin{description}
    \item[Caso de uso] Elegir campeonato
    \item[Descripción] El usuario desea seleccionar el campeonato que desea realizar.
    \item[Actores] Usuario
    \item[Precondiciones] El usuario ha elegido previamente en el menú principal una de las opciones: \textbf{campeonato} y ha
    elegido un personaje.
    \item[Postcondiciones] Selecciona un campeonato y se almacena en la configuración del juego.
    \item[Escenario principal] $\quad$
        \begin{enumerate}
            \item El usuario desea seleccionar el circuito en el que competir.
            \item El sistema muestra los campeonatos disponibles
            \item El usuario selecciona un campeonato y pulsa sobre la opción aceptar.
            \item El sistema almacena todos los circuitos del campeonato en la configuración del sistema.
            \item El sistema inicia el modo jugar campeonato.
        \end{enumerate}
    \item[Extensiones --- flujo alternativo] $\quad$
        \begin{description}
            \item[*a ] El usuario cierra la ventana de la aplicación y sale de la aplicación
            \item[*b ] El usuario pulsa la opción cancelar.
            \begin{enumerate}
                \item El sistema vuelve a la selección de personaje.
            \end{enumerate}
        \end{description}
\end{description}

\subsubsection{Caso de uso: Jugar carrera}

\begin{description}
    \item[Caso de uso] Jugar carrera
    \item[Descripción] El usuario juega una carrera
    \item[Actores] Usuario
    \item[Precondiciones] El usuario ha seleccionado un circuito o un campeonato.
    \item[Postcondiciones] El usuario completa una carrera.
    \item[Escenario principal] $\quad$
        \begin{enumerate}
            \item El sistema carga el circuito, el jugador, inteligencia artificial y los ítems.
            \item El sistema muestra la pantalla de juego.
            \item El usuario y el sistema interactúan durante la carrera.
            \item El usuario completa la carrera.
            \item El sistema muestra las posiciones finales de la carrera.
            \item El usuario pulsa continuar.
            \item El sistema pasa al menú principal
        \end{enumerate}
    \item[Extensiones --- flujo alternativo] $\quad$
        \begin{description}
            \item[*a ] El usuario cierra la ventana de la aplicación y sale de la aplicación
            %\item[*b] El usuario pulsa el botón de pausa.
             %   \begin{enumerate}
              %      \item El sistema muestra el menú de pausa.
               % \end{enumerate}
        \end{description}
\end{description}

\subsubsection{Caso de uso: Pausar}

\begin{description}
    \item[Caso de uso] Pausar
    \item[Descripción] El usuario selecciona pausar el juego y puede reanudarlo, reiniciarlo o volver al menú principal.
    \item[Actores] Usuario
    \item[Precondiciones] Se está jugando una carrera
    \item[Postcondiciones] Ninguna.
    \item[Escenario principal] $\quad$
        \begin{enumerate}
            \item El usuario pulsa el botón de pausa.
            \item El sistema detiene todos los elementos del juego y muestra el menú de pausa.
            \item El usuario pulsa la opción reanudar.
            \item El sistema reanudar la carrera.
        \end{enumerate}
    \item[Extensiones --- flujo alternativo] $\quad$
        \begin{description}
            \item[*a ] El usuario cierra la ventana de la aplicación y sale de la aplicación

            \item[*b] El usuario pulsa la opción reiniciar.
                \begin{enumerate}
                    \item El sistema reinicia la carrera.
                \end{enumerate}
                
            \item[*b] El usuario pulsa la opción menú.
                \begin{enumerate}
                    \item El sistema vuelve al menú principal.
                \end{enumerate}
        \end{description}
\end{description}

\subsubsection{Caso de uso: Lanzar ítem}

\begin{description}
    \item[Caso de uso] Salir
    \item[Descripción] El usuario lanza el ítem que tenga actualmente.
    \item[Actores] Usuario
    \item[Precondiciones] Se esta jugando una carrera y el usuario a recogido un ítem.
    \item[Postcondiciones] Se añade el ítem a la carrera.
    \item[Escenario principal] $\quad$
        \begin{enumerate}
            \item El usuario pulsa la opción de lanzar ítem.
            \item El sistema comprueba que el usuario posee un ítem y añade el ítem a la carrera.
        \end{enumerate}
    \item[Extensiones --- flujo alternativo] $\quad$
        \begin{description}
            \item[*a ] El usuario cierra la ventana de la aplicación y sale de la aplicación
        \end{description}
\end{description}

\subsubsection{Caso de uso: Mover vehículo}

\begin{description}
    \item[Caso de uso] Mover vehículo
    \item[Descripción] El usuario desplaza al vehículo por el circuito.
    \item[Actores] Usuario
    \item[Precondiciones] Se está jugando una carrera.
    \item[Postcondiciones] Ninguna.
    \item[Escenario principal] $\quad$
        \begin{enumerate}
            \item El usuario pulsa sobre una de las teclas de movimiento. W o Flecha hacia arriba, para mover el vehículo
            hacia adelante o S o Flecha hacia abajo, para moverlo marcha atrás.
            \item El sistema mueve al vehículo teniendo en cuenta su orientación, velocidad y ángulo.
            \item El sistema comprueba que no ha colisionado con ningún obstáculo u otro competidor.
        \end{enumerate}
    \item[Extensiones --- flujo alternativo] $\quad$
        \begin{description}
            \item[*a ] El usuario cierra la ventana de la aplicación y sale de la aplicación
            \item[1a ] El usuario pulsa las teclas de dirección a la vez que avanza. D o Flecha hacia la derecha, para girar
            a la derecha o, A o Flecha hacia la izquierda, para girar a la izquierda.
                \begin{enumerate}
                    \item El sistema gira al coche hacia la zona deseada.
                \end{enumerate}
            \item[3a ] El sistema detecta que ha colisionado con un competidor
            o con algún obstáculo.
                \begin{enumerate}
                    \item El sistema corrige la posición del vehículo.
                \end{enumerate}
        \end{description}
\end{description}

\subsubsection{Caso de uso: Jugar campeonato}

\begin{description}
    \item[Caso de uso] Jugar campeonato
    \item[Descripción] El usuario juega un campeonato.
    \item[Actores] Usuario
    \item[Precondiciones] El usuario selección previamente la opción \textbf{campeonato} en el menú principal.
    \item[Postcondiciones] Se juega un campeonato.
    
    \item[Escenario principal] $\quad$
        \begin{enumerate}
            \item El usuario desea jugar un campeonato
            \item El sistema carga el circuito actual
            \item El sistema y usuario interactúan en la carrera. Include Jugar carrera.
            \item Una vez terminada la carrera el sistema muestras las posiciones del campeonato.
            \item El usuario selección la opción continuar.
            \item El sistema pasar al siguiente circuito.
            \item Volveremos al punto 2.
        \end{enumerate}
    \item[Extensiones --- flujo alternativo] $\quad$
        \begin{description}
            \item[*a ] El usuario cierra la ventana de la aplicación y sale de la aplicación
            
            \item[2a ] Ya no hay más circuitos restantes.
                \begin{enumerate}
                    \item El sistema muestra la clasificación final de todos los jugadores del campeonato.
                    \item El usuario pulsa la opción continuar.
                    \item El sistema muestra la posición del jugador.
                \end{enumerate}
        \end{description}
\end{description}

\subsubsection{Caso de uso: Opciones}

\begin{description}
    \item[Caso de uso] Opciones
    \item[Descripción] El usuario desea modificar las opciones del juego.
    \item[Actores] Usuario
    
    \item[Precondiciones] El usuario seleccionó en el menú principal la opción \textbf{Opciones}.
    \item[Postcondiciones] El usuario modifica las opciones del juego.
    
    \item[Escenario principal] $\quad$
        \begin{enumerate}
            \item El usuario desea modificar las opciones del juego.
            \item El sistema muestra las opciones de juego.
            \item El usuario modifica las distintas opciones de juego.
            \item El usuario esta conforme con los cambios realizados y pulsa sobre el botón aceptar.
            \item El sistema almacena todos los cambios en la configuración.
            \item El sistema vuelve al menú principal.
        \end{enumerate}
    \item[Extensiones --- flujo alternativo] $\quad$
        \begin{description}
            \item[*a ] El usuario cierra la ventana de la aplicación y sale de la aplicación.
            \item[*b ] El usuario selecciona la opción cancelar.
                \begin{enumerate}
                    \item El sistema vuelve al menú principal.
                \end{enumerate}
            
            %\item [4a ] El usuario desea modificar las opciones de pantalla y pulsa sobre el boton de opciones de pantalla.
             %   \begin{enumerate}
              %      \item El sistema muestra las opciones de pantalla.
               %     \item El usuario modifica las distintas opciones de pantalla.
                %\end{enumerate}
            
            %\item [4b ] El usuario desea modificar las opciones de controles y pulsa sobre el boton de opciones de controles.
             %   \begin{enumerate}
              %      \item El sistema muestra las opciones de controles.
               %     \item El usuario modifica las distintas opciones de control.
                %\end{enumerate}
            
        \end{description}
\end{description}

\subsubsection{Caso de uso: Créditos}

\begin{description}
    \item[Caso de uso] Salir
    \item[Descripción] Se muestran la pantalla de créditos donde se reflejan los creadores del juego.
    \item[Actores] Usuario
    \item[Precondiciones] Ninguna.
    \item[Postcondiciones] Ninguna.
    \item[Escenario principal] $\quad$
        \begin{enumerate}
            \item El sistema muestra la pantalla de créditos.
            \item El usuario pulsa la opción volver.
            \item El sistema vuelve al menú principal.
        \end{enumerate}
    \item[Extensiones --- flujo alternativo] $\quad$
        \begin{description}
            \item[*a ] El usuario cierra la ventana de la aplicación y sale de la aplicación
        \end{description}
\end{description}

\subsubsection{Caso de uso: Salir}

\begin{description}
    \item[Caso de uso] Salir
    \item[Descripción] El usuario desea cerrar la aplicación.
    \item[Actores] Usuario
    \item[Precondiciones] Ninguna
    \item[Postcondiciones] Se sale de la aplicación.
    \item[Escenario principal] $\quad$
        \begin{enumerate}
            \item El usuario desea salir de la aplicación.
            \item El usuario pulsa la opción salir del menú principal.
            \item El sistema cierra la aplicación.
        \end{enumerate}
    \item[Extensiones --- flujo alternativo] $\quad$
        \begin{description}
            \item[*a ] El usuario cierra la ventana de la aplicación y sale de la aplicación
        \end{description}
\end{description}

%%%%%%%%%%%%%% MODELO CONCEPTUAL DE DATOS %%%%%%%%%%%%%%%%%%%5
\section{Modelo conceptual de datos}

\paragraph{}
Este apartado del análisis sirve para especificar los requisitos del sistema y las relaciones estáticas que
existen entre ellos.

\paragraph{}
Para este fin se utiliza como herramienta los diagramas de clase. En estos diagramas se representan
las clases de objetos, las asociaciones entre dichas clases, los atributos que componen las clases y las
relaciones de integridad.


\subsection{Diagrama de clases conceptuales}

\paragraph{}
En este apartado se muestra una lista con las diferentes clases necesarias para la realización de sistema. Junto a cada una de
las clases habrá una pequeña descripción sobre la labro que desempeña cada una.

\begin{description}
    \item [Juego] Clase principal de la aplicación, encargada de inicializar el sistema y el flujo entre unos apartados y otros.
    \item [Estado] Clase virtual, con las necesidades básicas de los estados del juego.
    
    \item [Menú Básico] Clase virtual, con las necesidades básicas de los menús.
    \item [Menú principal] Clase que gestiona el menú principal.
    \item [Menú selección personaje] Clase que gestiona el menú de selección de personaje.
    \item [Menú selección circuito] Clase que gestiona el menú de selección de circuito.
    \item [Menú opciones] Clase que gestiona el menú de opciones.
    \item [Menú de pausa] Clase que gestiona el menú de pausa.
    \item [Cursor] Cursor de los menús.
    \item [Botón] Clase que representa el botón en los menús.
    
    \item [Modo de juego] Clase virtual, con las necesidades básicas de los distintos modos de juego.
    \item [Carrera rápida] Clase que gestiona el modo de juego carrera rápida.
    \item [Campeonato] Clase que gestiona el modo de juego campeonato.
    \item [Contrarreloj] Clase que gestiona el modo de juego contrarreloj.
    
    \item [Control de juego] Clase encargada del control de la carrera, controlando la interacción del jugador con el circuito, así 
    como el jugador con los coches dirigidos por el ordenador. Aspectos básico como colisiones, scroll de pantalla, 
    control de vueltas, control
    de posiciones.
    
    \item [Circuito] Clase encargada de cargar y dibujar el circuito.
    \item [Gestor de colisiones] Clase encargada de detectar y gestionar las colisiones.
    
    \item [Objeto de juego] Clase virtual con las necesidades básicas de los objetos del juego.
    \item [Caja de ítem] Clase que representa las cajas que proporcionan ítems a los jugadores.
    \item [Vehículo básico] Clase virtual con las necesidades básicas de los vehículos del juego.
    \item [IA] Clase que representa el comportamiento de los vehículos dirigidos por el ordenador
    \item [Jugador] Clase que que representa al vehículo controlado por el jugador.
\end{description}

\paragraph{}
En las siguientes imágenes podemos ver los diagrama de clases asociado a los requisitos obtenidos. Por razones de espacio en el
documento y para una mejor apreciación de las clases necesarias, se ha visto conveniente dividir en dos partes diferenciadas
principalmente.

\paragraph{}
La primera figura muestra el primer diagrama las clases relacionadas con las pantalla de juego, como pueden ser menús y 
modos de juego.

\begin{figure}[H]
  \label{diagrama_clases_conceptuales}
  \begin{center}
    \includegraphics[scale=0.45]{imagenes/analisis/diagrama_clases_conceptuales2.png}
  \end{center}
  \caption{Análisis: Diagrama de clases conceptuales 1}
\end{figure}

\paragraph{}
En la segunda figura vemos el diagrama de clases referente a las clases principales que intervienen en la gestión de la pantalla
de juego.

\begin{figure}[H]
  \label{diagrama_clases_conceptuales}
  \begin{center}
    \includegraphics[scale=0.4, angle=90]{imagenes/analisis/diagrama_clases_conceptuales1.png}
  \end{center}
  \caption{Análisis: Diagrama de clases conceptuales 2}
\end{figure}

\section{Modelo de comportamiento del sistema}

\paragraph{}
El modelo de comportamiento especifica como debe actuar el sistema. El sistema es el que engloba todos los objetos, y el modelo
consta de dos partes:

\begin{itemize}
    \item Diagramas de secuencias del sistema: muestran la secuencia de eventos entre el usuario y el sistema.
    \item Contrato de las operaciones del sistema: describen el efecto que producen las operaciones en el sistema.
\end{itemize}

%%%%%%%%%%%%%%%%%%%%%%%%%%%%%%%%%%%%%%%%%%%%%%%%%%%%%%%%%%%%%%%%%%%%%%%%%%%%%%%%%%%%%%%%%%%%%%%%%%
%%%%%%%%%%%%%%%%%%%%%%%%%%%%%%%%%%%%%%%%%%%%%%%%%%%%%%%%%%%%%%%%%%%%%%%%%%%%%%%%%%%%%%%%%%%%%%%%%%
%%%%%%%%%%%%%%%%%%%%%%%%%%%%%%%%%%%%%%%%%%%%%%%%%%%%%%%%%%%%%%%%%%%%%%%%%%%%%%%%%%%%%%%%%%%%%%%%%%
%%%%%%%%%%%%%%%%%%%%%%%%%%%%%%%%%%%%%%%%%%%%%%%%%%%%%%%%%%%%%%%%%%%%%%%%%%%%%%%%%%%%%%%%%%%%%%%%%%
%%%%%%%%%%%%%%%%%%%%%%%%%%%%%%%%%%%%%%%%%%%%%%%%%%%%%%%%%%%%%%%%%%%%%%%%%%%%%%%%%%%%%%%%%%%%%%%%%%
%%%%%%%%%%%%%%%%%%%%%%%%%%%%%%%%%%%%%%%%%%%%%%%%%%%%%%%%%%%%%%%%%%%%%%%%%%%%%%%%%%%%%%%%%%%%%%%%%%
%%%%%%%%%%%%%%%%%%%%%%%%%%%%%%%%%%%%%%%%%%%%%%%%%%%%%%%%%%%%%%%%%%%%%%%%%%%%%%%%%%%%%%%%%%%%%%%%%%
\subsection{Diagramas de secuencia y contrato de las operaciones del sistema.}

\paragraph{}
No todos los posibles diagramas de secuencia aparecerán, nos centraremos en los más importantes, los que implican algún tipo de 
cambio en el sistema.

\subsubsection{Caso de uso: Menú principal(escenario principal)}

\begin{figure}[H]
  \label{secuencia_menu_principal1}
  \begin{center}
    \includegraphics[scale=0.7]{imagenes/analisis/secuencia_menu_principal1.png}
  \end{center}
  \caption{Análisis: Diagrama de secuencia Menú principal (escenario principal)}
\end{figure}

\begin{description}
    \item [Operación] InicioAplicacion()
    \item [Actores] Jugador, sistema.
    \item [Responsabilidades] inicia la aplicación y muestra el menú principal.
    \item [Precondiciones] Ninguna
    \item [Postcondiciones] $\quad$
        \begin{itemize}
            \item El sistema inicia todos los subsistemas necesarios para la correcta ejecución de la aplicación.
            \item El sistema crea un objeto con el Menú principal
        \end{itemize}
\end{description}

\begin{description}
    \item [Operación] ElegirPersonaje()
    \item [Actores] Jugador, Sistema.
    \item [Responsabilidades] salir del menú principal y acceder a la pantalla de elección de personaje.
    \item [Precondiciones] $\quad$
        \begin{itemize}
            \item Existe u objeto del menú principal.
        \end{itemize}
    \item [Postcondiciones] $\quad$
        \begin{itemize}
            \item Se destruye el objeto del menú principal.
        \end{itemize}
\end{description}

\subsubsection{Caso de uso: Menú principal(escenario 2a)}

\begin{figure}[H]
  \label{secuencia_menu_principal2}
  \begin{center}
    \includegraphics[scale=0.7]{imagenes/analisis/secuencia_menu_principal2.png}
  \end{center}
  \caption{Análisis: Diagrama de secuencia Menú principal (escenario 2a)}
\end{figure}

\begin{description}
    \item [Operación] SeleccionarOpciones()
    \item [Actores] Jugador, Sistema
    \item [Responsabilidades] salir del menú principal y entrar en el menú de opciones.
    \item [Precondiciones] $\quad$
        \begin{itemize}
            \item Existe un objeto del menú principal.
        \end{itemize}
    \item [Postcondiciones] $\quad$
        \begin{itemize}
            \item Se destruye el objeto del menú principal.
        \end{itemize}
\end{description}

\subsubsection{Caso de uso: Menú principal(escenario 2b)}

\begin{figure}[H] 
  \label{secuencia_menu_principal3}
  \begin{center}
    \includegraphics[scale=0.7]{imagenes/analisis/secuencia_menu_principal3.png}
  \end{center}
  \caption{Análisis: Diagrama de secuencia Menú principal (escenario 2b)}
\end{figure}

\begin{description}
    \item [Operación] SeleccionarCréditos()
    \item [Actores] Jugador, sistema.
    \item [Responsabilidades] salir del menú principal y entrar en la pantalla de créditos.
    \item [Precondiciones] $\quad$
        \begin{itemize}
            \item Existe un objeto del menú principal.
        \end{itemize}
    \item [Postcondiciones] $\quad$
        \begin{itemize}
            \item Se destruye el objeto del menú principal.
        \end{itemize}
\end{description}

\subsubsection{Caso de uso: Menú principal(escenario 2c)}

\begin{figure}[H] 
  \label{secuencia_menu_principal4}
  \begin{center}
    \includegraphics[scale=0.7]{imagenes/analisis/secuencia_menu_principal4.png}
  \end{center}
  \caption{Análisis: Diagrama de secuencia Menú principal (escenario 2c)}
\end{figure}

\begin{description}
    \item [Operación] SeleccionarSalir()
    \item [Actores] Jugador, sistema.
    \item [Responsabilidades] Salir de menú principal y salir de la aplicación
    \item [Precondiciones] $\quad$
        \begin{itemize}
            \item Existe un objeto del menú principal.
        \end{itemize}
    \item [Postcondiciones] $\quad$
        \begin{itemize}
            \item Se destruye el objeto del menú principal.
        \end{itemize}
\end{description}

\subsubsection{Caso de uso: Elegir personaje (escenario principal)}

\begin{figure}[H] 
  \label{secuencia_elegir_personaje}
  \begin{center}
    \includegraphics[scale=0.7]{imagenes/analisis/secuencia_elegir_personaje1.png}
  \end{center}
  \caption{Análisis: Diagrama de secuencia Elegir personaje (escenario principal)}
\end{figure}

\begin{description}
    \item [Operación] PantallaElegirPersonaje()
    \item [Actores] Jugador, sistema.
    \item [Responsabilidades] carga y muestra la pantalla de elección de personaje
    \item [Precondiciones] Ninguna
    \item [Postcondiciones] $\quad$
        \begin{itemize}
            \item Crea un objeto de la clase Menú personaje.
        \end{itemize}
\end{description}

\begin{description}
    \item [Operación] SeleccionarPersonaje()
    \item [Actores] Jugador, sistema.
    \item [Responsabilidades] marca un personaje como seleccionado
    \item [Precondiciones] $\quad$
        \begin{itemize}
            \item Existe un objeto del menú de personaje.
            \item Existe el personaje seleccionado.
        \end{itemize}
    \item [Postcondiciones] $\quad$
        \begin{itemize}
            \item Se marca el personaje como seleccionado.
            \item Se destruye el objeto del menú de personaje.
        \end{itemize}
\end{description}

\subsubsection{Caso de uso: Elegir circuito (escenario principal)}

\begin{figure}[H] 
  \label{secuencia_elegir_circuito}
  \begin{center}
    \includegraphics[scale=0.7]{imagenes/analisis/secuencia_elegir_circuito.png}
  \end{center}
  \caption{Análisis: Diagrama de secuencia Elegir circuito (escenario principal)}
\end{figure}

\begin{description}
    \item [Operación] PantallaElegirCircuito()
    \item [Actores] Jugador, sistema.
    \item [Responsabilidades] crea y muestra la pantalla de elección de circuito.
    \item [Precondiciones] Ninguna.
    \item [Postcondiciones] $\quad$
        \begin{itemize}
            \item Crea un objeto de la clase menú de circuito.
        \end{itemize}
\end{description}

\begin{description}
    \item [Operación] SeleccionarCircuito()
    \item [Actores] Jugador, sistema.
    \item [Responsabilidades] marca un circuito como seleccionado.
    \item [Precondiciones] $\quad$
        \begin{itemize}
            \item Existe el circuito seleccionado.
        \end{itemize}
    \item [Postcondiciones] $\quad$
        \begin{itemize}
            \item Se marca el circuito como seleccionado.
            \item Se destruye el objeto de menú de circuito.
        \end{itemize}
\end{description}

\subsubsection{Caso de uso: Elegir campeonato (escenario principal)}

\begin{figure}[H] 
  \label{secuencia_elegir_campeonato}
  \begin{center}
    \includegraphics[scale=0.7]{imagenes/analisis/secuencia_elegir_campeonato.png}
  \end{center}
  \caption{Análisis: Diagrama de secuencia Elegir campeonato (escenario principal)}
\end{figure}

\begin{description}
    \item [Operación] PantallaElegirCampeonato()
    \item [Actores] Jugador, sistema.
    \item [Responsabilidades] crea y muestra la pantalla de selección de campeonato
    \item [Precondiciones] Ninguna.
    \item [Postcondiciones] $\quad$
        \begin{itemize}
            \item Crea un objeto de la clase menú campeonato.
        \end{itemize}
\end{description}

\begin{description}
    \item [Operación] SeleccionarCampeonato()
    \item [Actores] Jugador, sistema.
    \item [Responsabilidades] selecciona un campeonato.
    \item [Precondiciones] $\quad$
        \begin{itemize}
            \item Existe el campeonato seleccionado.
        \end{itemize}
    \item [Postcondiciones] $\quad$
        \begin{itemize}
            \item Se marca el campeonato seleccionado.
            \item Se destruye el objeto de menú campeonato.
        \end{itemize}
\end{description}

\subsubsection{Caso de uso: Jugar carrera (escenario principal)}

\begin{figure}[H] 
  \label{secuencia_jugar}
  \begin{center}
    \includegraphics[scale=0.7]{imagenes/analisis/secuencia_jugar.png}
  \end{center}
  \caption{Análisis: Diagrama de secuencia Jugar carrera (escenario principal)}
\end{figure}

\begin{description} 
    \item [Operación] PantallaJuego()
    \item [Actores] Jugador, sistema.
    \item [Responsabilidades] carga y muestra la pantalla de jugo con el circuito seleccionado, inicia el juego.
    \item [Precondiciones] Ninguna
    \item [Postcondiciones] $\quad$
        \begin{itemize}
            \item Se crea un objeto de Control de juego.
            \item Se crea un objeto de Circuito
            \item Se crea un objeto de Jugador.
            \item Se crea un objeto de Gestor de colisiones.
            \item Se crea un objeto de IA por cada competidor de carrera.
            \item Se crea un objeto de Caja de ítem por cada caja del circuito.
        \end{itemize}
\end{description}

\begin{description}
    \item [Operación] Interactuar()
    \item [Actores] Jugador, sistema.
    \item [Responsabilidades] permite al jugador interactuar con el mundo 2D y los elementos que este posee.
    \item [Precondiciones] $\quad$
        \begin{itemize}
            \item Existe un objeto de Control de juego.
            \item Existe un objeto de Jugador.
        \end{itemize}
    \item [Postcondiciones] Ninguna.
\end{description}

\subsubsection{Caso de uso: Pausar (escenario principal)}

\begin{figure}[H] 
  \label{secuencia_pausar1}
  \begin{center}
    \includegraphics[scale=0.7]{imagenes/analisis/secuencia_pausar1.png}
  \end{center}
  \caption{Análisis: Diagrama de secuencia Pausar(escenario principal)}
\end{figure}

\begin{description}
    \item [Operación] PausarJuego()
    \item [Actores] Jugador, sistema.
    \item [Responsabilidades] pausa el juego y detiene todos los elementos de este.
    \item [Precondiciones] $\quad$
        \begin{itemize}
            \item Existe un objeto de Control juego.
            \item Se está jugando una carrera.
        \end{itemize}
    \item [Postcondiciones]
        \begin{itemize}
            \item Se crea un objeto de Menú de pausa.
        \end{itemize}
\end{description}

\begin{description}
    \item [Operación] ReanudarJuego()
    \item [Actores] Jugador, sistema.
    \item [Responsabilidades] reanuda la partida y quita el menú de pausa.
    \item [Precondiciones] $\quad$
        \begin{itemize}
            \item Existe un objeto de Control juego.
            \item La carrera estaba pausada
        \end{itemize}
    \item [Postcondiciones] $\quad$
        \begin{itemize}
            \item Se destruye el objeto del menú de pausa.
        \end{itemize}
\end{description}

\subsubsection{Caso de uso: Pausar (escenario *b)}

\begin{figure}[H] 
  \label{secuencia_pausar2}
  \begin{center}
    \includegraphics[scale=0.7]{imagenes/analisis/secuencia_pausar2.png}
  \end{center}
  \caption{Análisis: Diagrama de secuencia Pausar (escenario *b)}
\end{figure}

\begin{description}
    \item [Operación] ReiniciarJuego()
    \item [Actores] Jugador, sistema.
    \item [Responsabilidades] reinicia la carrera que se estaba jugando.
    \item [Precondiciones] $\quad$
        \begin{itemize}
            \item Existe un objeto de Control juego.
            \item La carrera estaba pausada
        \end{itemize}
    \item [Postcondiciones] $\quad$
        \begin{itemize}
            \item Se destruye el objeto del menú de pausa.
            \item Se reinicia la carrera.
        \end{itemize}
\end{description}

\subsubsection{Caso de uso: Pausar (escenario *c)}

\begin{figure}[H] 
  \label{secuencia_pausar3}
  \begin{center}
    \includegraphics[scale=0.7]{imagenes/analisis/secuencia_pausar3.png}
  \end{center}
  \caption{Análisis: Diagrama de secuencia (escenario )}
\end{figure}

\begin{description}
    \item [Operación] VolverMenuPrincipal()
    \item [Actores] Jugador, sistema.
    \item [Responsabilidades] se para la carrera jugada y se vuelve al menú principal.
    \item [Precondiciones] $\quad$
        \begin{itemize}
            \item Existe un objeto de Control juego.
            \item La carrera estaba pausada
        \end{itemize}
    \item [Postcondiciones] $\quad$
        \begin{itemize}
            \item Se destruye el objeto del menú de pausa.
        \end{itemize}
\end{description}

\subsubsection{Caso de uso: Lanzar ítem (escenario principal)}

\begin{figure}[H] 
  \label{secuencia_lanzar_item}
  \begin{center}
    \includegraphics[scale=0.7]{imagenes/analisis/secuencia_lanzar_item.png}
  \end{center}
  \caption{Análisis: Diagrama de secuencia Lanzar ítem (escenario principal)}
\end{figure}

\begin{description}
    \item [Operación] LanzarItem()
    \item [Actores] Jugador, sistema.
    \item [Responsabilidades] hace que el jugador lance el ítem que tiene en ese momento.
    \item [Precondiciones] $\quad$
        \begin{itemize}
            \item Existe un objeto de Control juego.
            \item Existe un objeto de jugador.
            \item El jugador posee un ítem.
        \end{itemize}
    \item [Postcondiciones] $\quad$
        \begin{itemize}
            \item Se crea un objeto ítem.
            \item Se añade el objeto ítem a Control de juego.
        \end{itemize}
\end{description}

\subsubsection{Caso de uso: Mover vehículo (escenario principal)}

\begin{figure}[H] 
  \label{secuencia_mover_vehiculo}
  \begin{center}
    \includegraphics[scale=0.7]{imagenes/analisis/secuencia_mover_vehiculo.png}
  \end{center}
  \caption{Análisis: Diagrama de secuencia Mover vehículo (escenario principal)}
\end{figure}

\begin{description}
    \item [Operación] MoverVehiculo()
    \item [Actores] Jugador, sistema.
    \item [Responsabilidades] mueve al personaje por el mundo 2D.
    \item [Precondiciones] $\quad$
        \begin{itemize}
            \item Existe un objeto de Control juego.
            \item Existe un objeto de jugador.        
        \end{itemize}
    \item [Postcondiciones] $\quad$
        \begin{itemize}
            \item Se modifica la posición del objeto jugador.
        \end{itemize}
\end{description}

\subsubsection{Caso de uso: Opciones (escenario principal)}

\begin{figure}[H] 
  \label{secuencia_opciones}
  \begin{center}
    \includegraphics[scale=0.7]{imagenes/analisis/secuencia_opciones.png}
  \end{center}
  \caption{Análisis: Diagrama de secuencia opciones (escenario principal)}
\end{figure}

\begin{description}
    \item [Operación] PantallaOpciones()
    \item [Actores] Jugador, sistema.
    \item [Responsabilidades] se crea y muestra la pantalla de opciones.
    \item [Precondiciones] Ninguna.
    \item [Postcondiciones] $\quad$
        \begin{itemize}
            \item Se crea un objeto de menú de opciones
        \end{itemize}
\end{description}

\begin{description}
    \item [Operación] ModificarOpciones
    \item [Actores] Jugador, sistema.
    \item [Responsabilidades] se modifican las opciones y se vuelve al menú principal.
    \item [Precondiciones] $\quad$
        \begin{itemize}
            \item Existe un objeto de menú de opciones
        \end{itemize}
    \item [Postcondiciones] $\quad$
        \begin{itemize}
            \item Se aplican los cambios de opciones.
            \item Se destruye el objeto de menú de opciones.
        \end{itemize}
\end{description}

\subsubsection{Caso de uso: Créditos (escenario principal)}

\begin{figure}[H] 
  \label{secuencia_creditos}
  \begin{center}
    \includegraphics[scale=0.7]{imagenes/analisis/secuencia_creditos.png}
  \end{center}
  \caption{Análisis: Diagrama de secuencia Créditos(escenario principal)}
\end{figure}

\begin{description}
    \item [Operación] PantallaCréditos()
    \item [Actores] Jugador, sistema.
    \item [Responsabilidades] crea y muestra la pantalla de créditos
    \item [Precondiciones] Ninguna.
    \item [Postcondiciones] $\quad$
        \begin{itemize}
            \item Se crea un objeto de Menú de créditos.
        \end{itemize}
\end{description}

\begin{description}
    \item [Operación] SeleccionarMenu()
    \item [Actores] Jugador, sistema.
    \item [Responsabilidades] destruye la pantalla de créditos y vuelve el menú principal.
    \item [Precondiciones] $\quad$
        \begin{itemize}
            \item Existe un objeto de Menú de créditos.
        \end{itemize}
    \item [Postcondiciones] $\quad$
        \begin{itemize}
            \item Se destruye el objeto de Menú de créditos.
        \end{itemize}
\end{description}
